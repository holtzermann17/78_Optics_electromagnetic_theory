\documentclass[12pt]{article}
\usepackage{pmmeta}
\pmcanonicalname{KleinGordonEquation}
\pmcreated{2013-03-22 17:55:11}
\pmmodified{2013-03-22 17:55:11}
\pmowner{invisiblerhino}{19637}
\pmmodifier{invisiblerhino}{19637}
\pmtitle{Klein-Gordon equation}
\pmrecord{12}{40412}
\pmprivacy{1}
\pmauthor{invisiblerhino}{19637}
\pmtype{Definition}
\pmcomment{trigger rebuild}
\pmclassification{msc}{78A25}
\pmclassification{msc}{35Q60}
\pmsynonym{Klein Gordon equation}{KleinGordonEquation}
\pmsynonym{Klein-Gordon-Fock equation}{KleinGordonEquation}
\pmrelated{DiracEquation}
\pmrelated{SchrodingersWaveEquation}

\endmetadata

% this is the default PlanetMath preamble.  as your knowledge
% of TeX increases, you will probably want to edit this, but
% it should be fine as is for beginners.

% almost certainly you want these
\usepackage{amssymb}
\usepackage{amsmath}
\usepackage{amsfonts}

% used for TeXing text within eps files
%\usepackage{psfrag}
% need this for including graphics (\includegraphics)
%\usepackage{graphicx}
% for neatly defining theorems and propositions
%\usepackage{amsthm}
% making logically defined graphics
%%%\usepackage{xypic}

% there are many more packages, add them here as you need them

% define commands here

\begin{document}
The Klein-Gordon equation is an equation of mathematical physics that describes spin-0 particles. It is given by:
\[
\Box \psi = \left(\frac{mc}{\hbar }\right)^2 \psi
\]
Here the $\Box$ symbol refers to the wave operator, or D'Alembertian, ($\Box = \nabla^2 - \frac{1}{c^2} \partial^2_t$)
 and $\psi$ is the wave function of a particle.
It is a Lorentz invariant expression.
\subsection{Derivation}
Like the Dirac equation, the Klein-Gordon equation is derived from the relativistic expression for total energy:
\[
E^2 = m^2c^4 + p^2c^2
\]
Instead of taking the square root (as Dirac did), we keep the equation in squared form and replace the momentum and energy with their operator equivalents, $E = i \hbar \partial_t$, $p = -i \hbar \nabla$. This gives (in disembodied operator form)
\[
-\hbar^2 \frac{\partial^2}{\partial t^2} = m^2 c^4 - \hbar^2 c^2 \nabla^2
\]
Rearranging:
\[
\hbar^2\left(c^2 \nabla^2 -\frac{\partial^2}{\partial t^2} \right)  = m^2 c^4
\]
Dividing both sides by $\hbar^2 c^2$:
\[
\left( \nabla^2 - \frac{1}{c^2}\frac{\partial^2}{\partial t^2} \right) =  \frac{m^2 c^2}{\hbar^2}
\]
Identifying the expression in brackets as the D'Alembertian and right-multiplying the whole expression by $\psi$ , we obtain the Klein-Gordon equation:
\[
\Box \psi = \left(\frac{mc}{\hbar }\right)^2 \psi
\]

%%%%%
%%%%%
\end{document}
